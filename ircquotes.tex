\documentclass[a4paper]{book}
\usepackage{hyperref}
\hypersetup{
    bookmarks=true,         % show bookmarks bar?
    unicode=false,          % non-Latin characters in Acrobat’s bookmarks
    pdftoolbar=true,        % show Acrobat’s toolbar?
    pdfmenubar=true,        % show Acrobat’s menu?
    pdffitwindow=false,     % window fit to page when opened
    pdfstartview={FitH},    % fits the width of the page to the window
    pdftitle={IRCQuotes from the Abyss of Ubuntu},    % title
    pdfauthor={Miia Ranta et al.},     % author
    pdfsubject={Ubuntu IRCQuotes},   % subject of the document
    pdfkeywords={ubuntu} {freenode} {irc} {quote}, % list of keywords
    pdfnewwindow=true,      % links in new window
    colorlinks=true,       % false: boxed links; true: colored links
    linkcolor=black,          % color of internal links
    citecolor=black,        % color of links to bibliography
    filecolor=black,      % color of file links
    urlcolor=blue           % color of external links
}

\title{IRCQuotes from the Abyss of Ubuntu}
\author{Miia Ranta \and operators and users of Ubuntu IRC channels}
\date{\today}

\begin{document}
\maketitle{}
\thispagestyle{empty}
\cleardoublepage{}

\pagestyle{headings}
\setcounter{page}{1}
\pagenumbering{roman}

\subsection*{Preface}
\addcontentsline{toc}{subsection}{Preface}

Ubuntu IRC channels in freenode thrive to be helpful, family-friendly 
and peaceful place to visit and get help for your problems related to 
Ubuntu Linux or just to relax with likeminded people. Discussion is quite 
fastpaced in the main channel of \verb$#ubuntu$, and can sometimes be quite 
quick in \verb$#ubuntu-offtopic$ as well, and special moments often go 
unobserved or they are forgotten quite soon after they've happened.

This collection of quotes, pamphlet or book, or a virtual photocopy, 
aims to be a collection of these special moments. Hopefully you'll 
have as much fun reading it as we've had living the moments and 
compiling them into this.
\\
\\
\today, Miia Ranta
\tableofcontents
\cleardoublepage

\setcounter{page}{1}
\pagenumbering{arabic}
\chapter{How it all began}

\section*{its a carrot only white}
\url{http://irclogs.ubuntu.com/2012/01/19/%23ubuntu.html#t22:31}

\begin{verbatim}
< shadaloo> ActionParsnip: hey it's trying to create an 8.5GB .iso
< shadaloo> How can I get a 4.6ishGB .iso? ^^
< ActionParsnip> shadaloo: did you select to fix the DVD size, it'll 
  reduce quality and such so it fits
< shadaloo> probably not
< shadaloo> ActionParsnip: where is that?
< shadaloo> I looked in adv. options
< shadaloo> didn't see
< shadaloo> ActionParsnip: where you at
< ActionParsnip> shadaloo: 
  http://imagecdn.maketecheasier.com/2009/03/devede.jpg
  note the "adjust disk image" button...
< shadaloo> what's a parsnip
< ActionParsnip> shadaloo: jesus child hav some patience
< ActionParsnip> shadaloo: i'm finding an image, I'm not that fast
< itaylor57> shadaloo, its a carrot only white
< shadaloo> I see
< shadaloo> hey ActionParsnip
< shadaloo> when I click that nothing happens
< shadaloo> I tried clicking the bar, also nothing
< ActionParsnip> shadaloo: do you have the right media size?
< ActionParsnip> shadaloo: nice lack of apology too, do you have any 
  manners?
< shadaloo> this family wrote a letter to my mom once saying how 
  exceptional my manners were
< shadaloo> I met them on a vacation
\end{verbatim}

This discussion made me wonder whatever happened to the quotes we Ubuntu
operators have collected over the sever or so years the channels we frequent
have existed. The quote isn't brilliant but you've got to start from somewhere, right?
\newpage{}

The next discussion followed elsewhere:
\begin{verbatim}
< Myrtti> oh man
< Myrtti> ActionParsnip is quote material sometimes
< Myrtti> I don't know if that was bad enough to tell him off for it
< AlanBell> we don't really have an organised quote file do we?
< Myrtti> could put something on an etherpad
< Myrtti> or similar
< Myrtti> I could compile a book and sell it for ONE MILLION DOLLAR
< Myrtti> FRIGGING SHARKS WITH LAZERS
< Myrtti> I could start a project in github
< Myrtti> do it with LaTeX...
< Myrtti> mmmm latex
< Myrtti> yes, I think I will.
< Myrtti> :-D
* AlanBell hugs Myrtti 
* Myrtti cracks her knuckles and launches her pink emacs
\end{verbatim}

And so this book, or file, or whatever you want to call it, began.
\newpage{}

\section*{constant buzz/hum I hear}

\url{http://irclogs.ubuntu.com/2012/01/20/%23ubuntu.html#t00:55}
\begin{verbatim}
< CT1> Hi all.  Is there a package (or a feature in audacity) that 
  plays a certain tone... changeable (subtly) via a GUI slider?  I'm 
  interested in finding out the frequency of the (yes, I may be 
  "mental") constant buzz/hum I hear.  It's essential in deciding if 
  I should be "committed" or if the tone is related to UK AC 
  frequency and its harmonics.
< genesisbot> lol CTI
< genesisbot> CT1*
< CT1> genesisbot: Thankfully the humorous side of my question was 
  duly noted.  The subtle hint of it being a real question was 
  perhaps lost in it's "insanity"
\end{verbatim}

The discussion that followed included suggestions of coding an app that
could do what was requested (a slider to adjust pitch) with PyGame, or
setting up an Arduino board with a physical knob. \verb$*shakes head*$
\end{document}
